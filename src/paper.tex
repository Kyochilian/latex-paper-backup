\documentclass[conference]{IEEEtran}   % 单栏会议模板;去 conference 选项即双栏
\usepackage[UTF8, scheme = plain]{ctex} % 中文支持
\usepackage[T1]{fontenc}
\usepackage{lmodern}

% ---------------- 常用 AI 宏包 ----------------
\usepackage{amsmath, amssymb, amsthm}
\usepackage{algorithm}
\usepackage{algpseudocode}   % 伪代码
\usepackage{graphicx}
\usepackage{booktabs}        % 三线表
\usepackage{multirow}
\usepackage{array}
\usepackage{subfigure}
\usepackage{hyperref}
\usepackage{cite}            % IEEE 引用风格
\usepackage{xcolor}
\usepackage{microtype}

% ---------------- 定理环境 ----------------
\newtheorem{theorem}{Theorem}
\newtheorem{lemma}{Lemma}
\newtheorem{remark}{Remark}

% ---------------- 自定义命令 ----------------
\newcommand{\ie}{\textit{i.e.}}
\newcommand{\eg}{\textit{e.g.}}
\newcommand{\etal}{\textit{et al.}}

% ---------------- 正文开始 ----------------
\begin{document}

\title{基于动态扫描机制的Vision RWKV让医学影像分割轻量而高效}

\author{%
  \IEEEauthorblockN{sora\IEEEauthorrefmark{1},
                    Zhendong Li\IEEEauthorrefmark{1},
                    }
  \IEEEauthorblockA{\IEEEauthorrefmark{1}Ningxia University\\
                    Email: kyochilian@gmail.com}
  \IEEEauthorblockA{\IEEEauthorrefmark{1}Ningxia University\\
                    Email: lizhendong@nxu.edu.cn}
}

\maketitle

\begin{abstract}
基于Transformer的分割框架是当前医学影像分割的主流方法,能够构建全局关系。
然而,在需要高精度与高分辨率的医学图像分割中,Transformer计算复杂度高,应用有限。
最近的RWKV模型降低了空间聚合复杂度并具备全局处理能力,但其在视觉领域存在长程建模羸弱的问题,同时循坏嵌套机制在分割领域中的解码性能不足。
为了解决这些问题,我们提出了一种基于Vision RWKV的分割框架---,该框架在具有较低复杂度的同时具备了全局建模能力。
为了增强RWKV的长程空间连续性,我们设计了一种动态扫描机制,该机制将图像分为全局block与局部block,根据不同层次的编码器动态地调整顺序和方向,
将全局与局部特征融合,大大提升了模型的精度。
同时,我们在解码器中构建了上采样模块---,该模块改良了RWKV的循环嵌套机制,提升了重建多尺度特征的能力。
我们还采用多种方法进一步降低---的计算复杂度,实现模型的轻量化。(LRFormer 低分辨率自注意力,池化QKV等等)
实验表明,我们的方法在多个数据集取得优异性能,同时计算复杂度与计算速度显著增强,使其在高精度分割的同时具备轻量化特性。
代码开源至:\url{https://github.com/shepherdxu/SCI-winning}。
\end{abstract}

\begin{IEEEkeywords}
computer vision, semantic segmentation, RWKV, Transformer, lightweight
\end{IEEEkeywords}

\section{Introduction}
\IEEEPARstart{S}{emantic} Segmentation 是计算机视觉中的一项重要任务,医学影像分割是关键应用之一,它将复杂的医学图像分割成不同的区域,
使器官、病灶及注意区域清晰可见,对于医学辅助诊断与治疗具有重要意义。
相较于人工,使用计算机视觉技术进行辅助诊断不仅提高了诊断效率,还提升了诊断精度,因此许多方法被提出用于医学影像分割任务。
传统的ML(机器学习)方法在研究中被广泛应用,如基于图割(Graph Cut)的方法~\cite{boykov2001interactive},随机森林(Random Forest)~\cite{geremia2011spatial}等。
近年来,随着FCN(Full Convolution Network)的提出~\cite{long2015fully},深度学习方法成为图像语义分割的主流方法。
随着Unet~\cite{ronneberger2015u}的提出,基于深度学习的分割方法占据了主导地位,通过上下采样,以及创新性地跳跃连接,使其成为了语义分割的主流框架。


写好文章介绍。首先写语义分割,然后写医学影像分割,然后写方法迭代。

neural networks (DNNs) have achieved remarkable success in various AI tasks.
However, their enormous computational cost hinders deployment on resource-limited devices.
Pruning redundant parameters is one of the most popular compression techniques~\cite{han2015deep}.

\section{Related Work}
\subsection{Magnitude-based Pruning}
Han \etal~\cite{han2015deep} proposed to remove weights with small absolute values.

\subsection{Dynamic Sparse Training}
Most recently, DST~\cite{evci2020rigging} allows the sparse topology to evolve during training.

\section{Methodology}
Let $\mathcal{W}=\{W_l\}_{l=1}^L$ denote the weights of an $L$-layer network.
Our goal is to find a sparse mask $M_l$ for each layer such that the remaining weights $W_l\odot M_l$ retain accuracy.

\subsection{Dynamic Growth Criterion}
We use the gradient magnitude as the saliency score:
\begin{equation}
s_{ij}^{(l)}=\left|\frac{\partial \mathcal{L}}{\partial w_{ij}^{(l)}}\right|.
\end{equation}


\section{Experiments}
We evaluate DST on CIFAR-10/100 and ImageNet with ResNet-50.

\begin{table}[h]
\centering
\caption{Top-1 accuracy (\%) on CIFAR-10 under different sparsities.}
\begin{tabular}{lcc}
\toprule
Method & 90\% sparsity & 95\% sparsity \\
\midrule
Baseline & 93.5 & 91.2 \\
Magnitude~\cite{han2015deep} & 92.8 & 89.7 \\
DST (ours) & \textbf{94.1} & \textbf{92.3} \\
\bottomrule
\end{tabular}
\end{table}

\section{Conclusion}
We presented a simple yet effective DST framework that dynamically adjusts sparse connectivity during training.
Future work includes extending DST to transformer architectures.

\section*{Acknowledgment}
This work was supported by the National Natural Science Foundation of China under Grant 62XXXXXX.

\bibliographystyle{IEEEtran}
\begin{thebibliography}{99}
\bibitem{han2015deep}
S.~Han, J.~Pool, J.~Tran, and W.~Dally,
``Learning both weights and connections for efficient neural network,''
\textit{Proc. NIPS}, 2015.

\bibitem{evci2020rigging}
U.~Evci, T.~Gale, J.~Menick, P.~S.~Castro, and E.~Elsen,
``Rigging the lottery: Making all tickets winners,''
\textit{Proc. ICML}, 2020.
\end{thebibliography}



\end{document}
% ================= end of main.tex =================