\documentclass[conference]{IEEEtran}   % 单栏会议模板;去 conference 选项即双栏
\usepackage[UTF8, scheme = plain]{ctex} % 中文支持
\usepackage[T1]{fontenc}
\usepackage{lmodern}

% ---------------- 常用 AI 宏包 ----------------
\usepackage{amsmath, amssymb, amsthm}
\usepackage{algorithm}
\usepackage{algpseudocode}   % 伪代码
\usepackage{graphicx}
\usepackage{booktabs}        % 三线表
\usepackage{multirow}
\usepackage{array}
\usepackage{subfigure}
\usepackage{hyperref}
\usepackage{cite}            % IEEE 引用风格
\usepackage{xcolor}
\usepackage{microtype}

% ---------------- 定理环境 ----------------
\newtheorem{theorem}{Theorem}
\newtheorem{lemma}{Lemma}
\newtheorem{remark}{Remark}

% ---------------- 自定义命令 ----------------
\newcommand{\ie}{\textit{i.e.}}
\newcommand{\eg}{\textit{e.g.}}
\newcommand{\etal}{\textit{et al.}}

% ---------------- 正文开始 ----------------
\begin{document}

\title{基于动态扫描机制的Vision RWKV让医学影像分割轻量而高效}

\author{%
  \IEEEauthorblockN{sora\IEEEauthorrefmark{1},
                    Zhendong Li\IEEEauthorrefmark{1},
                    }
  \IEEEauthorblockA{\IEEEauthorrefmark{1}Ningxia University\\
                    Email: kyochilian@gmail.com}
  \IEEEauthorblockA{\IEEEauthorrefmark{1}Ningxia University\\
                    Email: lizhendong@nxu.edu.cn}
}

\maketitle

\begin{abstract}

写好Transformer的不足 RNN的不足 写该工作的主要贡献 主要干了什么 解决了什么问题 代码在哪里能复现
\end{abstract}

\begin{IEEEkeywords}
deep learning, computer vision, segmentation, RWKV, RNN, Transformer
\end{IEEEkeywords}

\section{Introduction}
\IEEEPARstart{S}{emantic} Segmentation 是计算机视觉中的一项重要任务。写好文章介绍。首先写语义分割,然后写医学影像分割,然后写方法迭代。

neural networks (DNNs) have achieved remarkable success in various AI tasks.
However, their enormous computational cost hinders deployment on resource-limited devices.
Pruning redundant parameters is one of the most popular compression techniques~\cite{han2015deep}.

\section{Related Work}
\subsection{Magnitude-based Pruning}
Han \etal~\cite{han2015deep} proposed to remove weights with small absolute values.

\subsection{Dynamic Sparse Training}
Most recently, DST~\cite{evci2020rigging} allows the sparse topology to evolve during training.

\section{Methodology}
Let $\mathcal{W}=\{W_l\}_{l=1}^L$ denote the weights of an $L$-layer network.
Our goal is to find a sparse mask $M_l$ for each layer such that the remaining weights $W_l\odot M_l$ retain accuracy.

\subsection{Dynamic Growth Criterion}
We use the gradient magnitude as the saliency score:
\begin{equation}
s_{ij}^{(l)}=\left|\frac{\partial \mathcal{L}}{\partial w_{ij}^{(l)}}\right|.
\end{equation}


\section{Experiments}
We evaluate DST on CIFAR-10/100 and ImageNet with ResNet-50.

\begin{table}[h]
\centering
\caption{Top-1 accuracy (\%) on CIFAR-10 under different sparsities.}
\begin{tabular}{lcc}
\toprule
Method & 90\% sparsity & 95\% sparsity \\
\midrule
Baseline & 93.5 & 91.2 \\
Magnitude~\cite{han2015deep} & 92.8 & 89.7 \\
DST (ours) & \textbf{94.1} & \textbf{92.3} \\
\bottomrule
\end{tabular}
\end{table}

\section{Conclusion}
We presented a simple yet effective DST framework that dynamically adjusts sparse connectivity during training.
Future work includes extending DST to transformer architectures.

\section*{Acknowledgment}
This work was supported by the National Natural Science Foundation of China under Grant 62XXXXXX.

\bibliographystyle{IEEEtran}
\begin{thebibliography}{99}
\bibitem{han2015deep}
S.~Han, J.~Pool, J.~Tran, and W.~Dally,
``Learning both weights and connections for efficient neural network,''
\textit{Proc. NIPS}, 2015.

\bibitem{evci2020rigging}
U.~Evci, T.~Gale, J.~Menick, P.~S.~Castro, and E.~Elsen,
``Rigging the lottery: Making all tickets winners,''
\textit{Proc. ICML}, 2020.
\end{thebibliography}



\end{document}
% ================= end of main.tex =================